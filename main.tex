\documentclass[9pt,twocolumn,twoside]{rilabRxiv}
% Use the documentclass option 'lineno' to view line numbers
\setlength{\marginparwidth}{2cm}
\usepackage[textsize=tiny,colorinlistoftodos]{todonotes} % comments in margins
\definecolor{cornflowerblue}{rgb}{0.39, 0.58, 0.93}
\usepackage{blindtext}
\usepackage{nameref}
\raggedbottom
\usepackage{booktabs}
\usepackage{tabularx}
\usepackage{array}
\newcolumntype{Y}{>{\raggedright\arraybackslash}X}

%%%%%%%Add comments in color
\newcommand{\ms}[1]{{\small \textcolor{green}{#1}}}
\newcommand{\citex}[1]{{\small \textcolor{red}{CITE(#1)}}}
\newcommand{\X}{{\textcolor{red}{X}}}
\newcommand{\mex}{\textit{mexicana}\xspace}
\newcommand{\invfour}{\textit{Inv4m}\xspace}
\newcommand{\fdrgt} {$\textrm{\textit{FDR}} > 0.05$}
\newcommand{\fdreq} {$\textrm{\textit{FDR}} = 0.05$}
\newcommand{\fdrls} {$\textrm{\textit{FDR}} < 0.05$}
\newcommand{\parv}{\textit{parviglumis}\xspace}
\newcommand{\jmjii}{\textit{jmj2}\xspace}
\newcommand{\jmjiv}{\textit{jmj4}\xspace}
\newcommand{\jmjvi}{\textit{jmj6}\xspace}
\newcommand{\jmjix}{\textit{jmj9}\xspace}
\newcommand{\arabidopis}{\textit{Arabidopsis}\xspace}

\newcolumntype{b}{X}
\newcolumntype{s}{>{\hsize=.5\hsize}X}

% Set supplement numbers to S and start counting newly
\newcommand{\beginsupplement}{%
    \setcounter{table}{0}%
    \renewcommand{\tablename}{}%  % <-- 1. Make the word 'Table' blank
    \renewcommand{\thetable}{S\arabic{table} Table}%  % <-- 2. Include ' Table' after the number
    \setcounter{figure}{0}%
    \renewcommand{\figurename}{}%  % <-- Make the word 'Figure' blank for consistency
    \renewcommand{\thefigure}{S\arabic{figure} Fig}%  % <-- Include ' Figure' after the number
}

\usepackage{CJKutf8}
% \begin{CJK}{UTF8}{min}
% \verb|¯\_(ツ)_/¯|
% \end{CJK}

\title{Sorghum Lipid Database (SoLD): a database curation for lipid information for Sorghum Association Panel}

\author[$1$,$2$,*]{Nirwan Tandukar}
\author[$1$]{Ruthie Stokes}
\author[$3$]{Allison C. Barnes}
\author[$1$,*]{Rubén Rellán-Álvarez}
\affil[$1$,*]{Department of Molecular and Structural Biochemistry, N.C. Plant Sciences Initiative, North Carolina State University, Raleigh, NC, USA.}
\affil[$2$]{Genetics and Genomics Program, North Carolina State University, Raleigh, NC, USA}
\affil[$3$]{United States Department of Agriculture, Agricultural Research Service, Plant Science Research Unit, Raleigh, NC 27695}


\keywords{Sorghum Association Panel, Lipidomics, Genome-wide Association Studies, Random Forrest, Triglycerides, Phospholipids, Galactolipids}

\runningtitle{The Role of \textit{Invm4} in adaptation to low phosphorus availability} % For use in the footer

%% For the footnote.
%% Give the last name of the first author if only one author;
\runningauthor{Tandukar}
%% last names of both authors if there are two authors;
% \runningauthor{FirstAuthorLastname and SecondAuthorLastname}
%% last name of the first author followed by et al, if more than two authors.
\runningauthor{Tandukar \textit{et al.}}


%%% Abstract %%%%%%%%%%%%%%%%%%
\begin{abstract}
Sorghum lipidomics database
\end{abstract}


\setboolean{displaycopyright}{true}

\usepackage{hyperref}

\begin{document}

\maketitle
\thispagestyle{firststyle}
%\firstpagefootnote
\correspondingauthoraffiliation{
Department of Molecular and Structural Biochemistry, N.C. Plant Sciences Initiative, North Carolina State University, Raleigh, NC, USA.
E-mail: ntanduk@ncsu.edu, rrellan@ncsu.edu}
\vspace{-11pt}%

\setboolean{displaylineno}{true}
\ifthenelse{\boolean{displaylineno}}{\linenumbers}{}

\section{Introduction}

\lettrine[lines=2]{\color{color2}S}orghum is a plant. 


%%%%%%%%%%%%%%%%%%%%%%%%%%%%%%%%%%%%%%%%%%%%%%%%%%%%%%

\section{Results}

\subsection*{Lipidome overview under Control and Low-input field conditions}

We profiled early-stage sorghum lipid profiles from the SAP grown under two contrasting field conditions: (i) Control site and (ii) Low-input site, defined by reduced N and P availability and an earlier planting date to mimic cold stress. The overarching experimental workflow is outlined in Fig.~1A, including tissue collection and extraction, high-resolution LC-MS-based lipidomic profiling, GNPS-assisted lipid annotation, and post-acquisition data correction procedures through SERRF and SpATS implemented prior to downstream statistical and genetic analyses, as well as dissemination via the SoLD Shiny application (Fig.~1A).

Run-order quality control indicated stable analytical performance in both experimental conditions (Supplemental Fig.~S1A,B). Throughout the injection sequence, sample TICs conformed to expectations based on system suitability checks, internal standards, and blanks, and TIC variability among QC samples remained low (Control QC TIC CV = 7.5\%; LowInput QC TIC CV = 6.0\%; Supplemental Fig.~S1A,B). SERRF normalization enhanced technical precision in both environments, reducing the median QC-RSD from 3.6\% to 2.0\% in the Control condition and from 1.6\% to 0.3\% in the LowInput condition (Supplemental Fig.~S1C,D). Consistently, the proportion of features satisfying a QC-RSD $<30\%$ threshold increased from 75.7\% to 94.1\% in Control and from 77.3\% to 89.3\% in LowInput (Supplemental Fig.~S1C,D). In line with the reduction in technical variance, PCA of normalized data revealed a more compact clustering of samples relative to the unprocessed signal (Supplemental Fig.~S1E,F). Given that samples were derived from field-grown material, we further accounted for field-position effects using spatial modeling (SpATS) in R. Residual maps indicated that broad-scale row and column structures were largely mitigated following correction (Supplemental Fig.~S1G,H). Collectively, these procedures generated a normalized, spatially corrected lipid trait matrix that was used for all downstream analyses.


% =========================
% MAIN FIGURE 1
% =========================
\begin{figure*}[!ht]
  \centering
  \includegraphics[width=\textwidth]{figures/main/Figure1_Lipidomics_Landscape.png}
  \caption{\textbf{Lipidomics landscape under Control and LowInput field conditions.}
  (A) Field-to-database workflow. (B) Class-level CLR contrasts (LowInput--Control).
  (C) Class-average chemical shifts (weighted mean total carbons and double bonds).
  (D) LION enrichment summary (LowInput vs Control).}
  \label{fig:fig1_lipidomics_landscape}
\end{figure*}

At the lipid class level, the sorghum lipidome was predominantly composed of chloroplast galactolipids and major membrane phospholipids, with DG and TG constituting a substantial fraction of the neutral-lipid pool (Supplemental Fig.~S2A). Under Control conditions, MGDG accounted for 34.5\% of the mean TIC and declined to 32.1\% under LowInput, whereas PC remained comparatively stable (25.6\% in Control versus 24.8\% in LowInput). In contrast, TG increased from 2.7\% (Control) to 5.4\% (LowInput), and SQDG decreased from 2.9\% (Control) to 1.4\% (LowInput) (Supplemental Fig.~S2A). Compositional contrast analyses corroborated these trends as there was a marked relative depletion of SQDG and a pronounced positive contrast for PS (Fig.~1B), and TG-referenced ALR contrasts delineated complementary log-ratio shifts across lipid classes (Supplemental Fig.~S2B). Collectively, these findings indicate LowInput-associated lipid remodeling consistent with reduced sulfolipid abundance and increased partitioning into neutral TG lipids.

To characterize structural remodeling beyond total class abundance, we embedded each lipid class into a reduced chemical space defined by the weighted mean total carbon number and weighted mean double-bond count (Fig.~1C). TG occupied the high‑carbon region of this space and exhibited a pronounced shift toward higher total carbon under the LowInput condition, accompanied by a comparatively modest change in average unsaturation (Fig.~1C). The remaining classes maintained distinct, class‑specific carbon/unsaturation signatures, indicating that the effects of LowInput manifested both as shifts in class abundance (described above) and as coordinated, class‑level changes in acyl‑chain composition (Fig.~1C).

Across both environments, we quantified 244 distinct lipid molecular species, with substantial overlap between conditions (153 shared species) and smaller environment-specific subsets (49 Control-exclusive; 42 LowInput-exclusive; Supplemental Fig.~S3A). TG and PC represented the most species-rich classes in each environment (TG: 71 in Control, 69 in LowInput; PC: 39 in Control, 38 in LowInput), and broader class-level summaries indicated that glycerolipids and glycerophospholipids constituted the majority of identified species (Supplemental Fig.~S3B,D). PCA of lipid species further revealed that the multivariate structure of the dataset was strongly driven by lipid class, with TGs separating from membrane lipid classes and galactolipids forming distinct clusters under both environmental conditions (Supplemental Fig.~S4). 


Given that the SAP encompasses broad genetic diversity, we next evaluated lipid variance across genotypes and identified MGDG(18:3/18:3) as the highest-variance lipid in both environments (Supplemental Fig.~S5A,B). Seven of the ten highest-variance lipids were shared between Control and LowInput (Supplemental Fig.~S5C), suggesting that a conserved subset of lipid traits exhibits strong genotype sensitivity across environments. Within this high-variance subset, PC species were the most frequent class in both conditions (Control: 7/10; LowInput: 6/10; Supplemental Fig.~S5D), supporting their prioritization for subsequent genetic mapping and predictive modeling analyses.



\subsection*{Ratio-based multivariate analysis reveals coordinated lipid class transitions under Low-input}

To resolve \emph{relative} reallocations among major lipid pools, rather than absolute abundance differences that may be confounded by run-to-run intensity scaling, we aggregated individual molecular species into lipid classes (galactolipids, phospholipids, lysophospholipids, and neutral glycerolipids) and computed all pairwise class ratios on a log$_{10}$ scale. An OPLS-DA model constructed from these ratios clearly discriminated Control from LowInput samples along the predictive component (Fig.~3A), indicating that the LowInput lipidome is characterized by systematic \emph{rebalancing} among lipid classes. Model diagnostics indicated robust classification performance (1 predictive and 1 orthogonal component; $Q^2_{\mathrm{cum}}=0.977$, $R^2Y_{\mathrm{cum}}=0.977$, $R^2X_{\mathrm{cum}}=0.643$; Fig.~3D), and a label-permutation test (n=200) located the observed model performance at the extreme tail of the corresponding null distributions (Fig.~3C), consistent with a stable, non-overfitted multivariate pattern.

VIP-ranked ratios (VIP $>$ 1) revealed a highly structured pattern delineating which lipid pools most strongly discriminate Control from LowInput samples (Fig.~3B; Supplemental Fig.~S9A). First, many of the highest-ranking ratios were anchored on SQDG (e.g., MG/SQDG, PS/SQDG, PG/SQDG, PE/SQDG, PC/SQDG, LPE/SQDG), and these ratios were consistently elevated in LowInput (Supplemental Fig.~S9A), indicating a depletion of SQDG \emph{relative to multiple membrane-associated and neutral lipid pools}. This observation is concordant with the class-level profile, in which SQDG abundance decreases under LowInput conditions, whereas several other lipid classes (such as TG and, to a lesser extent, PS) exhibit relative enrichment (Fig.~1B; Supplemental Fig.~S2). Second, PS emerged as a central node in the remodeling landscape as multiple ratios with PS in the denominator (PC/PS, PE/PS, PG/PS, PA/PS, DG/PS, DGDG/PS, LPC/PS) were elevated in Control (Supplemental Fig.~S9A), implying that the LowInput condition is characterized by \emph{increased PS abundance relative to a broad spectrum of other lipid classes}. Collectively, the SQDG- and PS-centered ratios describe a LowInput-associated remodeling axis defined by (i) reduced representation of sulfolipids and (ii) a coordinated redistribution across the membrane lipid headgroup landscape.


% =========================
% MAIN FIGURE 2
% =========================
\begin{figure*}[!ht]
  \centering
  \includegraphics[width=\textwidth]{figures/main/Figure2_OPLS_DA.png}
  \caption{\textbf{OPLS-DA of summed lipid class ratios distinguishes Control and LowInput conditions.}
  (A) OPLS-DA scores plot showing separation of Control and LowInput samples along the predictive component (t$_1$), with the orthogonal component (to$_1$) capturing within-condition variation.
  (B) Top discriminating lipid class ratios ranked by variable importance in projection (VIP; VIP $>$ 1).
  (C) Permutation test (n=200) comparing observed $R^2_Y$ and $Q^2$ against the null distribution to evaluate model significance.
  (D) Model quality metrics summarizing the number of predictive/orthogonal components and cumulative $R^2_X$, $R^2_Y$, and $Q^2$.}
  \label{fig:Fig2_OPLS_DA}
\end{figure*}

Beyond these two anchors, the VIP set also highlighted interpretable ``transition-like'' relationships among neutral and membrane pools. Several MG-centered ratios increased in LowInput (MG/MGDG, MG/SQDG, MG/PG), while DG/MG and DGDG/MG were higher in Control (Supplemental Fig.~S9A), indicating that LowInput shifts the balance toward higher MG relative to both DG and galactolipids. In parallel, SQDG/TG was higher in Control (Supplemental Fig.~S9A), consistent with the expansion of TG observed in class composition summaries and the chemical-space displacement of TG under LowInput (Fig.~1C; Supplemental Fig.~S2). While ratio data do not directly measure metabolic flux, the \emph{direction-consistent} behavior of these ratios supports a coherent LowInput program in which plastid-associated glycolipids/sulfolipids are reduced relative to neutral storage and selected membrane pools.

To further motivate the use of ratios that capture condition-dependent \emph{rewiring} among lipid classes, we quantified partial correlations (controlling for total TIC) within each environment (Supplemental Fig.~S8). Numerous lipid-class pairs exhibited sign reversals in their partial correlations between Control and LowInput, indicating altered coordination among lipid pools under LowInput-associated stress. Ratios constructed from these sign-flip relationships displayed pronounced and directionally consistent shifts across conditions (Supplemental Fig.~S9B). Most significantly, TG increased in relative abundance compared with PE, MGDG, DG, and (in several cases) DGDG under LowInput (TG/PE, TG/MGDG, TG/DG; Supplemental Fig.~S9B), supporting a model of membrane-to-storage redistribution in which DG likely functions as a biosynthetic intermediate toward TG accumulation. In contrast, the TG/MG ratio was elevated in Control (Supplemental Fig.~S9B), indicating that MG increased disproportionately relative to TG under LowInput, consistent with the MG-centered VIP ratios described above (Supplemental Fig.~S9A). PS-centered partial-correlation ratios additionally supported a redistribution of headgroup classes. For example, PS/PC, PS/DG, and PS/MGDG were elevated in LowInput (Supplemental Fig.~S9B), paralleling the reciprocal pattern observed in the VIP analysis (PC/PS and DG/PS enriched in Control; Supplemental Fig.~S9A). Finally, lysophospholipid ratios indicated a reproducible shift in lyso-lipid balance as shown by LPC/LPE and LPC/PS being higher in Control, whereas LowInput favored LPE relative to multiple denominators (e.g., LPE/SQDG; LPE/MGDG; Supplemental Fig.~S9A). These patterns are consistent with stress-associated remodeling of phospholipid turnover pathways (e.g., PC→LPC and PE→LPE branches) rather than a uniform, condition-wide change in total lysolipid abundance (ref).

Thus, the ratio-based analyses converge on four biologically interpretable LowInput-associated lipid signatures that refine and extend the class-composition findings: (i) a broad depletion of SQDG relative to multiple lipid pools (consistent with the SQDG reduction in the global lipidome), (ii) a relative enrichment of PS within the phospholipid network (in line with the emergence of PS as a prominent class in compositional contrasts), (iii) an increase in TG relative to membrane lipids and DG (in line with the LowInput-linked expansion and chemical-space displacement of TG), and (iv) a reproducible lysophospholipid remodeling signal, with LPE increasing relative to multiple membrane pools and a reduced LPC/LPE balance, consistent with stress-responsive phospholipid turnover (PC$\rightarrow$LPC and PE$\rightarrow$LPE branches) rather than uniform scaling of total lyso-lipid abundance (ref). These coordinated patterns yield a concise representation of the lipid pools that increase relative to others and motivate targeted follow-up investigations into plastid membrane remodeling, phospholipid headgroup rebalancing, lysophospholipid turnover, and DG-to-TG conversion underlying storage lipid accumulation under LowInput field stress.


% =========================
% RESULTS (paste into Results)
% =========================


\subsection*{Genome-wide Association Studies of Lipid Traits Under LowInput}

We conducted GWAS using three representations of the same lipidomics dataset: (i) individual lipid species, (ii) summed abundances of lipid classes, and (iii) all possible pairwise ratios among these aggregated classes. We used a significance threshold of $p\le10^{-7}$ ($-\log_{10}(p)\ge 7$) and assigned SNPs to genes using a 25~kb window.

To identify and contextualize candidate genes, we used four complementary prioritization views: (1) genes with the highest occurrence across \emph{individual-lipid} GWAS phenotypes (Table~\ref{tab:LI_individual_gene_top}); (2) genes with the highest occurrence across \emph{summed-class and ratio} GWAS phenotypes (Table~\ref{tab:LI_sumratio_gene_top}); (3) lipid-class–focused GWAS (e.g., phospholipids and triacylglycerols) to prioritize class-relevant regulators; and (4) targeted follow-up of specific traits of interest, including sums (e.g., Sum~SQDG), biologically motivated ratios (cold-, nitrogen-, and phosphorus-linked), and individual compounds of interest (e.g., $\alpha$-carotene and zeaxanthin). We report LowInput results in the main text; full GWAS results and annotations for all traits are available in the Supplementary Tables and in the Shiny application.


\begin{table}[t]
\centering
\caption{Top LowInput genes from \textbf{individual-lipid} GWAS ranked by the maximum number of associated lipid phenotypes (25~kb mapping window; $p\le10^{-7}$). For each leading SNP, we show the single most recurrent gene to avoid redundant adjacent-gene listings at the same signal. Example lipids show a few representatives; full lists are in Supplementary Tables/Shiny.}
\label{tab:LI_individual_gene_top}
\scriptsize
\setlength{\tabcolsep}{3pt}
\renewcommand{\arraystretch}{1.08}
\begin{tabularx}{\columnwidth}{l r l l r Y}
\toprule
Gene & \#traits & Leading SNP & min $p$ & \#SNPs & Example lipids \\
\midrule
\texttt{SORBI\_3010G064200} & 20 & \texttt{SNP\_5083051} & $3.4\times10^{-16}$  & 8  & 3-Epilupeol; NAE(18:0); TG(10:0\_\allowbreak16:0\_\allowbreak16:0); TG(12:0\_\allowbreak14:0\_\allowbreak16:0); \ldots (20 total) \\
\texttt{SORBI\_3001G057100} & 16 & \texttt{SNP\_4303626} & $7\times10^{-67}$    & 20 & 13-Keto-9Z,11E-octadecadienoic acid; 13S-HOTrE; AEG(o-15:2\_\allowbreak15:0); DG(18:0\_\allowbreak20:3); \ldots (16 total) \\
\texttt{SORBI\_3003G169600} & 13 & \texttt{SNP\_29153705} & $7\times10^{-67}$    & 4  & AEG(o-15:2\_\allowbreak15:0); DG(18:0\_\allowbreak20:3); PC(18:1\_\allowbreak18:2); PC(18:1\_\allowbreak20:4); \ldots (13 total) \\
\texttt{SORBI\_3005G041700} & 13 & \texttt{SNP\_3875200}  & $7.7\times10^{-67}$  & 2  & AEG(o-15:2\_\allowbreak15:0); DG(18:0\_\allowbreak20:3); PC(18:1\_\allowbreak18:2); PC(18:1\_\allowbreak20:4); \ldots (13 total) \\
\texttt{SORBI\_3005G111700} & 13 & \texttt{SNP\_36855919} & $3.5\times10^{-13}$  & 14 & 13-Keto-9Z,11E-octadecadienoic acid; 13S-HOTrE; TG(12:0\_\allowbreak16:0\_\allowbreak16:0); TG(18:1\_\allowbreak20:3\_\allowbreak22:0); \ldots (13 total) \\
\texttt{SORBI\_3006G044532} & 12 & \texttt{SNP\_31061463} & $3.4\times10^{-16}$  & 3  & TG(12:0\_\allowbreak14:0\_\allowbreak16:0); TG(12:0\_\allowbreak16:0\_\allowbreak16:0); TG(12:0\_\allowbreak16:0\_\allowbreak18:3); \ldots (12 total) \\
\bottomrule
\end{tabularx}
\end{table}


\begin{table}[t]
\centering
\caption{Top LowInput genes from \textbf{summed-class and ratio} GWAS ranked by the maximum number of associated sum/ratio phenotypes (25~kb mapping window; $p\le10^{-7}$). For each leading SNP, we show the single most recurrent gene to avoid redundant adjacent-gene listings at the same signal. Example traits show a few representatives; full lists are in Supplementary Tables/Shiny.}
\label{tab:LI_sumratio_gene_top}
\scriptsize
\setlength{\tabcolsep}{3pt}
\renewcommand{\arraystretch}{1.08}
\begin{tabularx}{\columnwidth}{l r l l r Y}
\toprule
Gene & \#traits & Leading SNP & min $p$ & \#SNPs & Example sum/ratio traits \\
\midrule
\texttt{SORBI\_3008G088836} & 44 & \texttt{SNP\_21312400} & $2.1\times10^{-22}$  & 14 & Sum\_\allowbreak AEG/Sum\_\allowbreak Cer; Sum\_\allowbreak AEG/Sum\_\allowbreak DG; Sum\_\allowbreak AEG/Sum\_\allowbreak GalCer; Sum\_\allowbreak AEG/Sum\_\allowbreak SM; \ldots (44 total) \\
\texttt{SORBI\_3006G093600} & 37 & \texttt{SNP\_46347720} & $2.2\times10^{-14}$  & 11 & Sum\_\allowbreak AEG/Sum\_\allowbrea kFA; Sum\_\allowbreak AEG/Sum\_\allowbreak PG; Sum\_\allowbreak AEG/Sum\_\allowbreak PS; Sum\_\allowbreak AEG/Sum\_\allowbreak TG; \ldots (37 total) \\
\texttt{SORBI\_3004G127066} & 36 & \texttt{SNP\_15426192} & $7.7\times10^{-125}$ & 17 & Sum\_\allowbreak AEG/Sum\_\allowbreak Cer; Sum\_\allowbreak AEG/Sum\_\allowbreak DGDG; Sum\_\allowbreak AEG/Sum\_\allowbreak LPC; Sum\_\allowbreak AEG/Sum\_\allowbreak LPE; \ldots (36 total) \\
\texttt{SORBI\_3004G120000} & 35 & \texttt{SNP\_12933658} & $2.7\times10^{-115}$ & 16 & Sum\_\allowbreak AEG/Sum\_\allowbreak MG; Sum\_\allowbreak AEG/Sum\_\allowbreak MGDG; Sum\_\allowbreak AEG/Sum\_\allowbreak TG; Sum\_\allowbreak Cer; \ldots (35 total) \\
\texttt{SORBI\_3009G090800} & 33 & \texttt{SNP\_17785435} & $1.1\times10^{-301}$ & 15 & Sum\_\allowbreak AEG/Sum\_\allowbreak DGDG; Sum\_\allowbreakAEG/Sum\_\allowbreak TG; Sum\_\allowbreak Cer/Sum\_\allowbreak DGDG; Sum\_\allowbreakCer/Sum\_\allowbreak PA; \ldots (33 total) \\
\texttt{SORBI\_3009G086700} & 31 & \texttt{SNP\_14705908} & $4.2\times10^{-275}$ & 13 & Sum\_\allowbreak AEG/Sum\_\allowbreak DG; Sum\_\allowbreak AEG/Sum\_\allowbreak MG; Sum\_\allowbreak AEG/Sum\_\allowbreak MGDG; Sum\_\allowbreak AEG/Sum\_\allowbreak PA; \ldots (31 total) \\
\bottomrule
\end{tabularx}
\end{table}



\subsection{A DG-centered reaction subnetwork links membrane remodeling to TG accumulation under LowInput}

To connect the class-level shifts (SQDG depletion; TG enrichment; lysophospholipid remodeling; PS enrichment) with mechanistic lipid interconversions, we applied reaction-network enrichment (LINEX) and identified a compact DG-centered subnetwork that was preferentially enriched under LowInput (Fig.~5). The enriched module links several abundant DG species (DG(18:1/18:2), DG(16:0/18:2), DG(16:0/18:1)) to a key TG species (TG(16:0/18:1/18:2)) and multiple MG species (MG(16:0), MG(18:1), MG(18:2)), through reaction templates that collectively represent a membrane-to-storage redistribution axis. In particular, two reactions in the module couple TG formation to the *simultaneous* generation of lysophospholipids: (i) a PC–DG transacylation step producing TG and LPC (PC + DG $\leftrightarrow$ TG + LPC), and (ii) a PE–DG transacylation step producing TG and LPE (PE + DG $\leftrightarrow$ TG + LPE). Although LINEX annotates these templates using reference enzymes (e.g., “LRO1” or “PNPLA” labels), the biochemical interpretation in plants is most consistent with PDAT-like phospholipid:diacylglycerol acyltransferase activity (phospholipid $\rightarrow$ lysophospholipid while acylating DG to TG). The prominence of these edges provides a mechanistic explanation for the ratio-based signal showing lysophospholipid remodeling (e.g., LPE increasing relative to multiple pools and a shift in LPC/LPE balance) alongside TG accumulation.

The same subnetwork also contains a TG $\leftrightarrow$ DG reaction template (TG $\leftrightarrow$ DG), consistent with dynamic TAG turnover (lipase-mediated TG hydrolysis) coupled to re-esterification, as well as a DG + MG $\leftrightarrow$ TG template that highlights MG/DG intermediacy in glycerolipid remodeling. Together, these reactions place DG as a central hub that can be (i) generated from membrane lipid remodeling, (ii) funneled into storage TG, or (iii) recycled through TG turnover—exactly the type of coordinated remodeling expected under combined nutrient limitation and early-season stress.

Importantly, the enriched network aligns with candidate genes emerging from our GWAS analyses. The DG-to-TG conversion implied by the DG–TG edges is supported by identification of \textit{DGAT1} (\texttt{SORBI\_3010G170000}) across multiple TG GWAS traits, consistent with its known role catalyzing the terminal acylation of DG to form TG. :contentReference[oaicite:0]{index=0}  In addition, DG kinase genes (DGK2/3/5) were also highlighted among candidates for multiple DG- and TG-related summed/ratio traits, supporting a competing DG fate into PA that can modulate stress signaling and the balance between membrane remodeling and TG storage. :contentReference[oaicite:1]{index=1}  Upstream of these enzymatic steps, a P starvation regulatory node (\textit{PHR1}-like; \texttt{SORBI\_3001G384300}) was repeatedly associated with multiple PC and PE traits, consistent with phosphate-starvation–linked phospholipid remodeling that would feed into the PC/PE $\rightarrow$ lyso-lipid branches seen in the network. :contentReference[oaicite:2]{index=2}  Finally, the network-level emphasis on sulfolipid loss is concordant with a sulfate-assimilation candidate identified for the \emph{sum SQDG} GWAS (\textit{APK3}; \texttt{SORBI\_3005G195600}), providing genetic support for the SQDG depletion axis observed in the global lipidome and compositional contrasts. :contentReference[oaicite:3]{index=3}

Overall, the LINEX-enriched subnetwork provides a mechanistically coherent bridge between (i) the compositional/ratio signatures (SQDG depletion, TG enrichment, and lyso-lipid remodeling) and (ii) GWAS-supported candidate regulators at key branch points, highlighting DG as a metabolic hub and pointing to PDAT/DGAT- and DGK-mediated control of membrane-to-storage lipid reallocation under LowInput field stress.

% =========================
% FIGURE 5 caption block (Overleaf)
% =========================
\begin{figure*}[t]
  \centering
  \includegraphics[width=\textwidth]{figures/main/Figure5_LINEX_Enriched_Subnetwork.png}
  \caption{\textbf{LINEX-enriched lipid reaction subnetwork highlights DG-centered remodeling consistent with TG accumulation and lysophospholipid formation.}
  Circles denote lipid species (colored by class), and squares denote reaction/enzyme templates. The enriched module connects multiple DG species to TG through reaction templates representing DG$\rightarrow$TG formation (DGAT/PDAT-like), TG$\leftrightarrow$DG turnover (lipase/esterase-like), and phospholipid-linked transacylation steps that generate LPC or LPE as by-products (PC + DG $\leftrightarrow$ TG + LPC; PE + DG $\leftrightarrow$ TG + LPE).}
  \label{fig:Fig5_LINEX_Subnetwork}
\end{figure*}


\subsection*{Lipid modules predict plant height with high accuracy and cross-dataset reproducibility}

To identify lipid features predictive of agronomic traits, we applied XGBoost regression with SHAP-based interpretation. Given the high correlation among lipid species within biochemical classes, we first grouped the 226 individual lipid variables into 27 correlation-derived modules using hierarchical clustering based on Spearman correlation distances (Methods). This module-level strategy mitigates the recognized limitation whereby SHAP importance values become attenuated and unstable in the presence of highly collinear predictors.

\paragraph{Plant height prediction.}
For plant height, the XGBoost model exhibited robust predictive performance on our dataset ($R^2 = 0.752 \pm 0.027$, 5-fold CV; $n = 388$; Fig.~4A). Module-level SHAP analysis indicated that a single TG-dominated module (M1; 22 lipids) contributed the majority of the predictive signal, with a mean absolute module SHAP approximately 10-fold higher than the second-ranked module (Fig.~4B). Lipids assigned to M1 consisted primarily of short- and medium-chain triglycerides, including TG(10:0/10:0/10:0), TG(12:0/12:0/14:0), and TG(10:0/12:0/16:0). Although these individual lipids were associated with high SHAP values, they are statistically interchangeable as members of the same highly correlated module.

To further substantiate these findings, we replicated the analysis using plant height phenotypes obtained from an independent study (Boyles et al.). Although the overall predictive performance was reduced in this dataset ($R^2 = 0.36 \pm 0.154$; $n = 264$), the ranking of module importance was highly concordant i.e., 8 of the top 10 modules were shared between datasets, and the correlation of module importance scores was effectively perfect ($r = 0.99$; Fig.~4C,D). Notably, M1 (TG-enriched) remained the top-ranked module in both datasets, indicating that short-chain triglycerides represent robust and reproducible biomarkers of plant height across independent studies.

\paragraph{Flowering time shows no lipid-based predictive signal but reveals consistent module patterns.}
In contrast to plant height, lipid modules exhibited negligible predictive power for flowering time (days to anthesis). Across all five environments evaluated, including our DTA measurements and the four environments reported in Wei et al.\ (MI20, NE20, IA21\_1, IA21\_2), the cross-validated $R^2$ values were negative or close to zero (range: $-0.39$ to $-0.05$; Fig.~S11B), indicating that lipid profiles do not explain appreciable variation in flowering time.

Despite the absence of predictive signal, we evaluated whether any lipid modules exhibited consistent importance across environments. Four modules were recurrently ranked within the top 10 in all five environments: M1 (22 lipids, TG-dominated), M2 (22 lipids, PC-dominated), M5 (15 lipids, PC-dominated), and M6 (15 lipids, TG-dominated) (Fig.~S11A,D). Two additional modules (M3, M13) were among the top 10 in four of the five environments. This cross-environment consistency indicates that, although these lipid programs do not predict flowering time with sufficient accuracy for practical applications, they likely exhibit weak yet reproducible associations with developmental timing.


% =========================
% FIGURE PLACEHOLDERS
% =========================

\begin{figure*}[t]
  \centering
  \includegraphics[width=\textwidth]{Figure4_RF_SHAP_PlantHeight.png}
  \caption{\textbf{Random forest prediction of plant height and SHAP-based interpretation.}
  (A) Observed vs.\ predicted residualized plant height (RF). (B) Top 20 SHAP-ranked lipids.
  (C) Overlap of top 20 SHAP lipids between our dataset and Boyles et~al.
  (D) Cross-study correlation of SHAP importance for the top predictors.}
  \label{fig:fig4_rf_shap_height}
\end{figure*}

\begin{figure*}[p]
  \centering
  \includegraphics[width=\textwidth]{SuppFig_S10_RF_SHAP_details_PlantHeight.png}
  \caption{\textbf{RF/SHAP modeling details for plant height.}
  (A) Raw vs.\ PC-residualized plant height distribution. (B) Class composition of the top 50 SHAP-ranked lipids.
  (C) Marginal correlation with phenotype vs.\ mean SHAP importance. (D) 5-fold CV metrics across iterations.}
  \label{fig:suppfig_s10_rf_shap_details}
\end{figure*}

\begin{figure*}[p]
  \centering
  \includegraphics[width=\textwidth]{SuppFig_S11_FloweringTime_Environment_Comparison.png}
  \caption{\textbf{Flowering time prediction across environments.}
  (A) SHAP values of top lipids across environments. (B) Test-set correlation by environment.
  (C) Consistency of top SHAP lipids across environments. (D) Lipid prevalence across environments.}
  \label{fig:suppfig_s11_flowering_time}
\end{figure*}




\section{Discussion}

From our multi-omics analysis, we can infer that the maize phosphorus starvation response is shaped by leaf developmental stage, with older leaves showing enhanced stress responses indicative of the onset of developmental senescence during the vegetative phase.
While phosphorus deficiency triggered canonical molecular responses across genotypes, the magnitude of these responses varied depending on the leaf developmental position.
The \invfour chromosomal inversion showed minimal modulation of phosphorus starvation responses, indicating that its contribution to highland adaptation operates through effects on developmental timing rather than enhanced nutrient stress tolerance.


\subsection*{Candidate Gene Identification using Genes with Highest Occurrences in Lipid GWAS}

\paragraph{I. Candidate Gene Identification using Genes with Highest Occurrences in Individual GWAS}
Upon tabulation of all genes across all genotypes, the gene \texttt{SORBI\_3010G064200} exhibited the highest frequency of occurrence. This gene has a close homolog in rice, designated as GSL5, and in \emph{Arabidopsis thaliana}, designated as CalS5. Callose, a $\beta$-1,3-glucan, is a dynamic cell wall polysaccharide that plays a critical role in plant development, particularly during microsporogenesis. In rice, GLUCAN SYNTHASE-LIKE 5 (GSL5) has been identified as an essential callose synthase responsible for the deposition of callose in the primary cell walls of meiocytes and in the newly formed cell plates of tetrads \citep{Shi2015}. Loss-of-function mutations in GSL5 result in defective callose walls surrounding microspores, leading to male sterility, aberrant exine formation, and reduced pollen viability \citep{Shi2015}.

This functional role is highly conserved in its \emph{Arabidopsis} ortholog, AtGSL2 (also referred to as CalS5), which similarly regulates callose deposition in the cell walls of meiocytes, tetrads, and microspores, and is indispensable for proper pollen wall patterning and reproductive fertility \citep{Dong2005}. Moreover, ectopic or experimental expression of CalS5 has been shown to induce artificial callose accumulation in cell walls, thereby modifying cell wall permeability and mechanical properties \citep{Xie2012}.

An important environmental dimension has been documented in rice, where callose synthase expression in anthers is strongly chilling-responsive, indicating a molecular link between cold stress, callose metabolism, and the onset of male sterility \citep{Yamaguchi2006}. Collectively, these observations emphasize the conserved and central role of GSL5/AtGSL2 in regulating callose dynamics required for successful plant reproduction, while also highlighting the sensitivity of this pathway to environmental perturbations such as low temperature.

\paragraph{II. Candidate Gene Identification using Genes with Highest Occurrences in Sum and Ratio GWAS}
Regarding the sums and ratios of lipids, one of the most highly represented genes was \texttt{SORBI\_3004G120000} (Supp Table). Expansins constitute a superfamily of cell wall–loosening proteins, classified into α-expansins (EXPAs), β-expansins (EXPBs), and the expansin-like A (EXLA) and B (EXLB) subfamilies (Lee et al., 2001, p. 1; Choi et al., 2006, p. 1, 3). These proteins mediate pH-dependent extension of the plant cell wall and overall cell growth by disrupting non-covalent bonds between cell wall polysaccharides, thereby enabling turgor-driven polymer creep without detectable hydrolysis of cell wall polymers (Lee et al., 2001, p. 1, 2, 4). Through this mode of action, expansins directly modulate cell wall architecture, influencing processes such as fruit softening and the crystalline cellulose fraction (Lee et al., 2001, p. 5; Choi et al., 2006, p. 6).

In addition, expansins play a pivotal role in plant adaptive responses to environmental stresses, with their expression associated with acclimation mechanisms under submergence, ethylene exposure, O\textsubscript{2} deficiency, and drought-induced cell wall folding and desiccation tolerance (Lee et al., 2001, p. 3; Choi et al., 2006, p. 5–6). Expansin gene expression is also finely regulated by nitrogen availability, largely through cytokinin-dependent signaling pathways. Under low nitrate supply, reduced cytokinin levels promote primary root elongation, which correlates with elevated expression of expansin (EXP) and xyloglucan endotransglucosylase/hydrolase (XTH) genes (Teixeira et al., 2024, p. 1, 8). In contrast, high nitrate availability typically elevates cytokinin concentrations and suppresses primary root elongation (Teixeira et al., 2024, p. 1).

Selenium (Se) further modulates this regulatory network. Under low nitrate conditions, Se application increases root cytokinin concentrations and inhibits root growth by downregulating EXP and XTH gene expression, an effect associated with enhanced shoot sugar accumulation. Under high nitrate, Se exerts a negative effect on root cytokinin biosynthesis, which in turn upregulates EXP and XTH transcripts and results in increased root size (Teixeira et al., 2024, p. 8, 9). Collectively, these findings indicate that Se modifies the nitrate–cytokinin regulatory balance to adjust the partitioning of growth between shoot and root tissues.

\subsection{Candidate Gene Identification using lipid classes}

\paragraph{I. Phospholipid GWAS Identifies a Phosphate Starvation Response Gene}
Candidate genes were prioritized by analyzing GWAS results within the phospholipid lipid class. The Myb-like DNA-binding domain gene \texttt{SORBI\_3001G384300} was consistently identified (Fig \ref{fig:Fig3}A). This gene exhibits homology with \textit{PHR1} (PHOSPHATE STARVATION RESPONSE 1) in rice, a principal regulator of phosphate homeostasis. It demonstrated associations with several phospholipid traits, namely PC(16:0/20:3), PC(16:0/22:5), PC(16:0/22:6), PC(18:1/20:1), PE(16:0/18:1), PC(18:1/24:1), PC(18:2/20:0), and PC(18:3/0:0), indicating a prospective correlation between phospholipid remodeling and phosphate starvation signaling (Supp Table).

\textit{PHR1} serves as a central transcription factor within plants, balancing the responses to phosphate (Pi) deprivation. It is categorized under the MYB-CC family of transcription factors and demonstrates a high level of conservation across both vascular plants and unicellular algae (Rubio et al., 2001). \textit{PHR1} exhibits specific binding affinity to a cis-regulatory element termed the P1BS (GNATATNC) motif, which resides in the promoters of numerous genes induced by Pi starvation, thereby facilitating their expression in conditions of Pi deficiency. These genes encompass those that encode phosphate transporters, signaling components, and enzymes that partake in metabolic adaptations to Pi scarcity (Bustos et al., 2010). Loss-of-function phr1 mutants display compromised expression of genes responsive to Pi starvation and a diminished accumulation of anthocyanins, starch, and sugars under conditions of Pi deficiency, along with modified Pi distribution between roots and shoots (Rubio et al., 2001; Bustos et al., 2010). In contrast, overexpression of PHR1 results in augmented Pi uptake and improved responses to Pi starvation (Nilsson et al., 2007). In addition to maintaining phosphorus homeostasis, \textit{PHR1} also plays a crucial role in regulating sulfate homeostasis, particularly under conditions of phosphate deficiency. It enhances the expression of the sulfate transporter gene SULTR1;3 and influences the translocation of sulfate from the aerial parts to the roots during phosphorus starvation. The observation that mutants in either \texttt{phr1} or \texttt{sultr1;3} demonstrate diminished sulfate transfer from shoots to roots suggests that PHR1 is integral to the interaction and coordinated regulation of P and sulfur homeostasis (Rouached et al., 2011).


%References

%Rubio V, Linhares F, Solano R, Martín AC, Iglesias J, Leyva A, Paz-Ares J. (2001). A conserved MYB transcription factor involved in phosphate starvation signaling both in vascular plants and in unicellular algae. Genes & Development, 15(16), 2122–2133.

%Bustos R, Castrillo G, Linhares F, Puga MI, Rubio V, Pérez-Pérez J, Solano R, Leyva A, Paz-Ares J. (2010). A central regulatory system largely controls transcriptional activation and repression responses to phosphate starvation in Arabidopsis. PLoS Genetics, 6(9), e1001102.

%Nilsson L, Muller R, Nielsen TH. (2007). Increased expression of the MYB-related transcription factor, PHR1, leads to enhanced phosphate uptake in Arabidopsis thaliana. Plant Cell and Environment, 30(11), 1499–1512.

%Rouached H, Secco D, Arpat AB, Poirier Y. (2011). The transcription factor PHR1 plays a key role in the regulation of sulfate shoot-to-root flux upon phosphate starvation in Arabidopsis. BMC Plant Biology, 11, 19.

%Puga MI, Mateos I, Charukesi R, Wang Z, Franco-Zorrilla JM, de Lorenzo L, Irigoyen ML, Masiero S, Bustos R, Rodríguez J, Leyva A, Rubio V, Sommer H, Paz-Ares J. (2014). SPX1 is a phosphate-dependent inhibitor of PHOSPHATE STARVATION RESPONSE 1 in Arabidopsis. Proceedings of the National Academy of Sciences of the United States of America, 111(41), 14947–14952.

\paragraph{II. DGAT1 Controls Triacylglycerol Storage in Response to Nitrogen Limitation and Cold}
We identified the gene \texttt{SORBI\_3010G170000}, which encodes Acyl‐CoA:diacylglycerol acyltransferase 1 (DGAT1, analogous to Arabidopsis TG1), in five distinct GWASs: TG(18:1/18:3/22:0), TG(18:1/20:3/22:0), TG(18:2/18:2/18:4), TG(18:2/20:3/22:0), and TG(18:3/18:3/18:3) (Fig \ref{fig:Fig3}B). \textit{DGAT1} is responsible for the essential final conversion of DG into TG, which is a key lipid for carbon and energy storage in seeds and stress-affected vegetative tissues \cite{Zhang2009,Yang2011}. In Arabidopsis, low N levels result in TG accumulation within leaves due to increased levels of \textit{DGAT1} and \textit{OLEOSIN1} \cite{Yang2011}. The ABA signaling pathway, involving the transcription factor \textit{ABI4}, directly stimulates \textit{DGAT1} by interacting with CE1 elements (CACCG) in its promoter. In \emph{abi4} mutants, both \textit{DGAT1} stimulation and TG accumulation are reduced, emphasizing the significance of \textit{ABI4} during N deficiency \cite{Yang2011}. Additionally, \textit{DGAT1} is highly responsive to cold temperatures (4°C) and plays an essential role in freeze tolerance. Arabidopsis mutants deficient in \emph{dgat1} develop chlorosis and increased cell mortality under cold stress, with reduced TG but higher DG and PA levels \cite{Tan2018}. This elevated PA production induces RbohD-dependent ROS formation, causing oxidative stress. Increased DG kinase activity (DGK2/3/5) (also a candidate in our GWAS results for Sum\_AEG/Sum\_DG, Sum\_Cer/Sum\_DG, Sum\_DG/Sum\_TG, Sum\_DGDG/Sum\_TG, Sum\_MG/Sum\_TG, Sum\_MGDG/Sum\_TG; Supp Table 7) further boosts PA (also observed in our dataset, Fig. \ref{fig:Fig1_lipid_class}), while the removal of \emph{dgk} genes restores cold tolerance, suggesting a balance between \textit{DGAT1} and \textit{DGK} is essential for managing ROS and adapting to cold stress \cite{Tan2018}. In seeds, both \textit{DGAT1} and phospholipid:diacylglycerol acyltransferase 1 (\textit{PDAT1}) are vital for optimal oil body development. \emph{dgat1} mutants have a 20–40\% decline in seed oil content (see Lipid annotation Section 5), whereas double mutants (\emph{dgat1/pdat1}) or RNAi lines demonstrate an 80\% decrease in TG, resulting in fertility and embryonic issues \cite{Zhang2009}. Overexpression of \textit{DGAT1} enhances seed weight and oil production, highlighting its crucial role in regulating TG levels throughout plant development \cite{Zhang2009,Yang2011}.

\subsection{Candidate Gene Identification using Sum and Ratios of Lipids}

\paragraph{I. Sum of SQDG GWAS Identifies a Sulphate Assimilation Gene}
Our GWAS for the sum of SQDG species identified a sulfate assimilation gene called the adenylyl-sulfate kinase gene (\textit{APK3}, \texttt{SORBI\_3005G195600}). \textit{APK3} is one of the four isoforms of APS kinase (adenosine 5'-phosphosulfate kinase) in \textit{Arabidopsis thaliana}, an enzyme that plays a critical role in sulfur metabolism by phosphorylating adenosine 5'-phosphosulfate (APS) to produce 3'-phosphoadenosine 5'-phosphosulfate (PAPS), the active sulfate donor required for sulfation reactions in secondary metabolism (Mugford et al., 2009, p. 1–2). Unlike the other APK isoforms, \textit{APK3} is uniquely localized in the cytosol, whereas \textit{APK1}, \textit{APK2}, and \textit{APK4} are plastid-localized (Mugford et al., 2009, p. 4). The enzyme's activity influences sulfur partitioning between primary and secondary metabolism, particularly affecting the synthesis of sulfated secondary metabolites such as glucosinolates, which are important for plant defense (Mugford et al., 2009, p. 1–2). Studies have shown that disruption of \textit{APK1} and \textit{APK2} leads to a significant reduction in glucosinolate levels and an increase in thiols, indicating that APKs regulate the availability of PAPS and thus control the flux toward secondary sulfated compounds (Mugford et al., 2009, p. 2–3). However, the specific disruption of \textit{APK3}, the cytosolic isoform, does not significantly affect primary sulfate assimilation or glucosinolate levels, suggesting a more specialized or possibly redundant role compared to plastidic APKs (Mugford et al., 2009, p. 4). Overall, \textit{APK3} contributes to sulfur metabolism by modulating PAPS production in the cytosol, influencing sulfur flux balancing, but its precise regulatory role requires further elucidation.

%References:

%Mugford, S.G., Lee, B.-R., Koprivova, A., Matthewman, C.A., and Kopriva, S. (2009). Disruption of adenosine-5′-phosphosulfate kinase in Arabidopsis reduces levels of sulfated secondary metabolites. Plant Cell 21, 910–927. (Sum_SQDG_APK3_secondary_S.pdf, pp. 1–5)
%Kopriva, S., Mugford, S.G., Baraniecka, P., Lee, B.R., Matthewman, C.A., and Koprivova, A. (2012). Control of sulfur partitioning between primary and secondary metabolism in Arabidopsis. Frontiers in Plant Science, 3, 163. (Sum_SQDG_APK3_S_starvation.pdf, p.14)
%Takahashi, H. (2019). Sulfate transport systems in plants: functional diversity and molecular mechanisms underlying regulatory coordination. J. Exp. Bot. 70, 4075–4087. (Sum_SQDG_APK3_S_starvation.pdf, p.16) 

\paragraph*{II. Lipid Ratios identifies a Cold stress gene}
Calcium-dependent protein kinases (CDPKs) function as key regulators within plant signal transduction networks, modulating Ca$^{2+}$-dependent signaling cascades that are essential for plant growth, development, and adaptation to diverse environmental stimuli \cite{Abbasi_2004, Ye_2009}. Our findings corroborate a critical role for CDPKs in orchestrating cold stress responses in rice. In particular, the rice CDPK gene \textit{OsCDPK13} displayed pronounced accumulation in the leaf sheath of 2-week-old seedlings and in callus tissue, and its protein kinase activity in cytosolic extracts from leaf sheaths increased markedly following cold treatment at $5\,^{\circ}\mathrm{C}$ \cite{Abbasi_2004}. This cold-induced activation was accompanied by a coordinated up-regulation of \textit{OsCDPK13} transcript and protein levels, which reached maximal induction at $5\,^{\circ}\mathrm{C}$ and became detectable within 3 hours after the onset of cold stress \cite{Abbasi_2004}.

The direct contribution of \textit{OsCDPK13} to cold tolerance has been further corroborated by functional analyses. Transgenic rice lines overexpressing sense \textit{OsCDPK13} exhibited significantly higher recovery rates following exposure to low-temperature stress than vector control plants, indicating enhanced stress resilience \cite{Abbasi_2004}. This protective function is aligned with the observation that \textit{OsCDPK13} transcript accumulation is greater in cold-tolerant rice cultivars than in cold-sensitive ones \cite{Abbasi_2004}. Concordant results have been reported for other CDPK family members, such as \textit{OsCPK7} (also designated as \textit{OsCDPK13} in certain studies), which display elevated transcript levels in rice seedlings subjected to cold and salt stress; moreover, overexpression of these kinases confers broad-spectrum tolerance to cold, drought, and salinity \cite{Ye_2009}. Collectively, these findings identify \textit{OsCDPK13} as a pivotal signaling node that orchestrates rice acclimation to cold stress through coordinated regulation of gene expression, modulation of protein activities, and consequent physiological adjustments.

\paragraph*{III. Lipid Ratios identifies a Nitrogen stress gene}
Tryptophan (Trp), an aromatic amino acid, functions as an essential substrate for protein biosynthesis and as a pivotal signaling molecule in plants, where it contributes significantly to growth, development, and adaptation to a wide range of environmental stresses, including the regulation of osmotic homeostasis \citep{Wang2025}. It is a central metabolite and the principal precursor of indole-3-acetic acid (IAA), the predominant endogenous auxin in plants \citep{Fu2022, Liu2024}. The Trp-dependent pathway is regarded as the major route for IAA biosynthesis, which is indispensable for coordinating root cell division and elongation, hypocotyl elongation, and leaf expansion \citep{Fu2022, Liu2024}. Accordingly, exogenous Trp supplementation has been demonstrated to promote plant height, root system elongation, and overall vegetative growth, while concurrently enhancing plant tolerance to various abiotic and biotic stresses, in part through the modulation of carbon (C) and nitrogen (N) metabolism \citep{Fu2022}.

Under nitrogen-limiting conditions, tryptophan (Trp) metabolism exhibits significant alterations, directly influencing plant adaptive responses. For example, low-N tolerant sorghum genotypes exhibit higher endogenous Trp concentrations than sensitive genotypes, and transcriptome analyses have revealed distinct transcriptional reconfiguration of genes involved in the Trp metabolic pathway \citep{Liu2024}. Exogenous Trp application significantly elevates both Trp and indole-3-acetic acid (IAA) levels in sorghum roots under low-nitrogen stress, thereby promoting root system development and enhancing key physiological processes associated with carbon and N metabolism, which collectively confer improved low-nitrogen tolerance \citep{Liu2024}. Consistently, Trp also contributes to the mitigation of other abiotic stresses, such as salt–alkali stress in soybean, by enhancing photosynthetic performance and modulating sucrose–starch metabolism, ultimately resulting in increased yield \citep{Wang2025}. The Tryptophan Aminotransferase of Arabidopsis 1 (TAA1, also referred to as SAV3), a key enzyme in Trp-dependent IAA biosynthesis, has been identified as a central regulator of chlorophyll accumulation and N assimilation, thereby underscoring Trp’s role as a master regulatory node responsive to nitrogen availability \citep{Fu2022}. Collectively, these findings highlight the pivotal function of Trp in integrating root growth and stress tolerance through its regulation of IAA homeostasis and the coordinated balance of carbon and N metabolism.

\paragraph*{IV. Lipid Ratios identifies a Phosphorus stress gene}
Plant phosphate transporters (PTs), particularly members of the Pht1 family, play a pivotal role in phosphorus (P) homeostasis by mediating the active uptake of inorganic phosphate (Pi) from the soil and facilitating its subsequent translocation within the plant (Jia et al., 2011, p. 1; Li et al., 2015, p. 1). This transporter family is indispensable for the efficient remobilization of internal Pi from source to sink organs, a process that is critical for optimizing plant growth and development, particularly under P-deficient conditions (Li et al., 2015, p. 1, 7). For example, OsPht1;8 (OsPT8) in rice (\emph{Oryza sativa}) has been characterized as a high-affinity PT that actively contributes to Pi acquisition and long-distance transport from roots to shoots (Jia et al., 2011, p. 1, 7). The expression of OsPht1;8 has been detected in multiple tissues, including roots, shoots, and seeds, underscoring its broad physiological significance across diverse developmental stages (Jia et al., 2011, p. 1, 2).

The function of phosphate transporters (PTs) is tightly integrated with the plant’s adaptive responses to phosphorus (P) deficiency. Under inorganic phosphate (Pi)-limiting conditions, plants typically remobilize Pi from mature, senescing leaves to sustain metabolically active young tissues, a process critically mediated by transporters such as OsPT8 (Li et al., 2015, pp. 1, 5, 7). Our analysis demonstrated that OsPT8 transcript abundance in roots is markedly induced under Pi starvation (Jia et al., 2011, p. 2; Li et al., 2015, p. 7). Targeted knockdown of OsPT8 in rice shoots disrupted internal P redistribution, resulting in excessive P retention in older leaf blades and decreased P accumulation in younger blades when external Pi supply was limiting (Li et al., 2015, pp. 1, 5). In contrast, constitutive overexpression of OsPT8 caused pronounced Pi hyperaccumulation across plant tissues, leading to Pi toxicity symptoms under high-Pi conditions (Jia et al., 2011, pp. 1, 4). Collectively, these data highlight the essential role of Pht1 family transporters, particularly OsPT8, in controlling Pi uptake and internal redistribution, thereby maintaining P homeostasis under fluctuating environmental Pi availability.

%References (as provided):
%Nitrogen_deficiency.pdf pp. 6–9; membrane_remodeling_phosphorus.pdf pp. 237–243; membrane_lipid_P_reuse.pdf pp. 13–14; glycerolipid_remodeling_P_starve.pdf p. 7; glycolipid_remodeling_nitrogen_phosphorus_deficiency.pdf p. 13; Cold_tolerance_barley.pdf pp. 7–9; Glycerolipid_freezing.pdf pp. 4–8; Photosynthesis_thylakoid_glycerolipid.pdf pp. 4–7; Cold_tolerance_maize.pdf pp. 8–9.
%Nussberger et al., 1993; Hagio et al., 2002; Wada & Murata, 2007; Murata, 1983; Roughan, 1985; Murata & Yamaya, 1984; Murata et al., 1992; Wolter et al., 1992; Moon et al., 1995; Ishizaki-Nishizawa et al., 1996; Wu & Browse, 1995; Barkan et al., 2006; Gao et al., 2015; Gao et al., 2020.

\subsection*{Candidate Gene Identification using Individual Lipid}

\paragraph*{I. Alternative Oxidase Roles in Photoprotection and Nitrate Assimilation}
Through our GWAS focused on alpha-carotene, we identified an alternative oxidase (AOX) gene. $\alpha$-carotene, a secondary chloroplast carotenoid, is primarily located within the reaction centers of photosystem I (PSI) and photosystem II (PSII), with only minor quantities found in the peripheral light-harvesting complexes \citep{Young1989}. It bears structural similarity to $\beta$-carotene, absorbs blue-green light, and facilitates energy transfer to chlorophyll while concurrently quenching triplet chlorophyll and reactive oxygen species (ROS) to safeguard the photosynthetic apparatus from photooxidative damage under intense light stress. Its co-localization with $\beta$-carotene in pigment–protein complexes indicates a contributory role in stabilizing the core structures of PSI and PSII \citep{Young1989}. The mitochondrial AOX pathway offers a non-phosphorylating alternative to cytochrome oxidase, directly oxidizing ubiquinol to water, thereby preventing over-reduction of the photosynthetic electron transport chain \citep{Vishwakarma2015}. AOX1A, the dominant isoform in green tissues, plays a role in dissipating excess reducing equivalents produced by photosynthesis, supports non-photochemical quenching (NPQ), and collaborates with the chloroplast malate–oxaloacetate shuttle to sustain cellular redox homeostasis. Under conditions of stress, such as high light or drought, that inhibit the cytochrome pathway, AOX activity curbs ROS formation and maintains photosynthetic efficiency \citep{Vishwakarma2015}. In addition to its photoprotective function, AOX is also vital for nitrate assimilation in plants. During NO$_3^-$ reduction, the accumulation of reducing equivalents may lead to chloroplast over-reduction. AOX counters this by channeling excess reductants into mitochondrial respiration, thereby preventing oxidative stress and sustaining photosynthesis \citep{Gandin2014}. Studies involving \textit{aox1a} T-DNA insertion mutants in \emph{Arabidopsis thaliana} corroborate that AOX engages with nitrate assimilation pathways to uphold redox balance and optimize C-N metabolism under varying N conditions \citep{Gandin2014,Vishwakarma2015}.

\paragraph{II. GWAS of Zeaxanthin Reveals High Light-inducible Protein.}
Zeaxanthin is an essential carotenoid that plays a significant role in the photoprotection mechanisms of photosynthetic organisms, predominantly acting within the framework of the xanthophyll cycle. Under conditions of high light (HL) stress, violaxanthin undergoes enzymatic de-epoxidation to form antheraxanthin, which is further converted into zeaxanthin. This carotenoid is instrumental in dissipating excess excitation energy by quenching excited chlorophyll molecules. The process effectively averts the generation of deleterious reactive oxygen species (ROS), thus safeguarding PSII from photoinhibition (Levin and Schuster 2023). Zeaxanthin associates with light-harvesting complexes, such as LHCII and certain LHC-like proteins, thereby facilitating non-photochemical quenching (NPQ) to efficiently transmute excess absorbed photonic energy into thermal energy (Levin and Schuster 2023).

Our GWAS for zeaxanthin has identified the gene \texttt{SORBI\_3002G033800}, which has an orthologous counterpart in \emph{Arabidopsis}, referred to as One-helix proteins (OHPs). These OHPs also share homology with the high light-inducible proteins (HLIPs) found in cyanobacteria. These function as small chlorophyll a/b-binding proteins characterized by a single transmembrane helix with an LHC motif. OHPs are upregulated under high light conditions, playing a pivotal role in the biogenesis and repair of PSII. They transiently associate with PSII core proteins and temporarily bind chlorophyll pigments during the PSII repair cycle, shielding chlorophyll molecules from photooxidative damage by facilitating energy dissipation through the direct transfer between chlorophyll a and $\beta$-carotene (Levin and Schuster 2023). In \emph{Arabidopsis}, mutations in OHP1 result in compromised chlorophyll accumulation, thylakoid architecture, and photosystem functionality, highlighting their essential role in photoprotection and photosynthetic efficiency (Levin and Schuster 2023).

%References:
%Levin, G., & Schuster, G. (2023). LHC-like Proteins: The Guardians of Photosynthesis. International Journal of Molecular Sciences, 24, 2503. 12568911
%Levin, G., Yasmin, M., Simanowitz, M.C., Meir, A., Tadmor, Y., Hirschberg, J., Adir, N., & Schuster, G. (2022). A Desert Green Alga That Thrives at Extreme High-Light Intensities Using a Unique Photoinhibition Protection Mechanism. bioRxiv. 911
%Myouga, F., Takahashi, K., Tanaka, R., Nagata, N., Kiss, A.Z., Funk, C., Nomura, Y., Nakagami, H., Jansson, S., and Shinozaki, K. (2018). Stable accumulation of photosystem II requires ONE-HELIX PROTEIN1 (OHP1) of the light harvesting-like family. Plant Physiology, 176(4), 2277–2291.
%Hey, D., and Grimm, B. (2018). ONE-HELIX PROTEIN2 (OHP2) is required for the stability of OHP1 and assembly factor HCF244 and is functionally linked to PSII biogenesis. Plant Physiology, 177(4), 1453–1472.



\subsection*{Biological interpretation of TG in plant height}
The predominance of triglyceride modules in plant height prediction is biologically well substantiated. Triglycerides constitute the principal form of energy storage in seeds and vegetative tissues, and their mobilization supplies both carbon skeletons and metabolic energy required for growth processes. In particular, short-chain triglycerides may serve as indicators of metabolic flux through fatty acid biosynthetic pathways that support rapid biomass accumulation. The finding that these associations are consistently observed across independent datasets strengthens confidence that they represent true underlying biological relationships rather than artifacts of model overfitting.

The absence of robust lipid-based predictors for flowering time is itself informative. Floral transition is predominantly regulated by genetic determinants and environmental cues (e.g., photoperiod, vernalization) acting through well-characterized signaling networks. Although lipid metabolic processes intersect with these regulatory pathways, our findings indicate that steady-state lipid profiles obtained at a single developmental stage are insufficient to capture the underlying regulatory complexity governing flowering time specially because the lipid profiles were samples at the XXX stage. 





\section*{Conclusion}

Our multi-omics analysis reveals that the maize phosphorus starvation response is influenced by the leaf developmental stage during the vegetative phase, with older leaves positioned below the collar exhibiting enhanced stress responses characteristic of developmental senescence, which integrate nutrient limitation with natural developmental progression.
The bifurcation of molecular responses into light-harvesting shutdown and senescence acceleration shows coordinated regulation of functionally distinct pathways.
Lipidomic patterns parallel transcriptomic responses, with age-dependent amplification of phospholipid degradation, galactolipid accumulation, and triacylglycerol synthesis.
The divergence of lysophosphatidylethanolamine patterns from 16:3 plant models highlights the importance of lipid metabolism architecture differences between 16:3 and 18:3 plant species.
Despite this strong developmental dependency, the \invfour chromosomal inversion does not substantially modulate phosphorus starvation responses, indicating that its contribution to highland adaptation operates through constitutive effects on developmental timing rather than enhanced nutrient stress tolerance.


\section*{Materials and methods}

\subsection*{Plant Material and Growth Conditions}

In our study, we used the Sorghum Association Panel (SAP), consisting of 400 accessions designed to cover extensive genetic and phenotypic diversity. This collection includes both temperate-adapted breeding lines and tropical landraces. The panel represents five botanical races, bicolor, caudatum, durra, guinea, and kafir, capturing a diverse range of domestication and adaptation processes.

SAP was first genotyped using simple sequence repeat markers, followed by low-coverage genotyping by sequencing (GBS). For a more comprehensive variation set, Boatwright et al. (ref) resequenced all entries using whole genome sequencing (WGS) with an average depth of 38× (ranging from 25–72×). The variant data from WGS revealed approximately 43.98 million polymorphisms, including roughly 38 million SNPs with 5 million small insertions/deletions. While GBS variants were predominantly located in genic regions, the WGS data were more evenly distributed across genic and intergenic regions. Genome-wide linkage disequilibrium is approximately 20 kb, although there were deviations specific to each chromosome. The consequent high-density variant map establishes the resequenced SAP as a valuable tool for examining diversity and conducting genome wide association studies (GWAS).

We evaluated SAP accessions across two different field settings during two growing seasons (2019 and 2022) at the Pee Dee Research and Education Center, Clemson University, Florence, South Carolina. The "control" condition, herein denoted as C, involved standard agronomic inputs with sufficient levels of nitrogen (N) and phosphorus (P) along with a typical planting schedule. In contrast, the "low input" condition, herein denoted as LI,  featured reduced N and P coupled with earlier planting to mimic a cold stress environment.

\subsection*{Lipidomics Analysis}

\paragraph{Sample Preparation and Extraction.} 
Samples were prepared following the standard extraction protocols explained in the hpc1 paper (ref).


\paragraph{Liquid Chromatography-Mass Spectrometry (LC-MS) Analysis.} Lipid extracts were subjected to high-resolution mass spectrometry employing both positive and negative ionization modes to achieve comprehensive lipid coverage. Chromatographic separation was executed utilizing either reversed-phase columns, such as C18, or hydrophilic interaction chromatography (HILIC) columns, contingent upon the polarity of the lipids. In the case of C18 methodologies, the gradient elution commenced at 1 minute and concluded at 8 minutes, succeeded by an isocratic elution phase from 8 to 9.5 minutes. Data preceding 1 minute and subsequent to 9.5 minutes were omitted to eliminate solvent front and late eluting artifacts. For HILIC methodologies, the gradient initiation occurred at 1 minute, terminating at 16.25 minutes, and was followed by an isocratic elution extending until 18.5 minutes; data before 1 minute and after 18.5 minutes were excluded in a similar manner.

\paragraph{Feature Detection and Data Processing.} Raw LC-MS data, in both positive and negative ion modes, were processed utilizing MZmine 2, an open-source software for the analysis of mass spectrometry data (ref). The pipeline encompassed peak detection, chromatogram construction, deconvolution, and isotope filtering, producing a detailed feature table containing mass-to-charge ratio-retention time (mz-rt) pairs. Isotopic peaks were excluded to minimize redundancy, as a single metabolite can yield multiple co-eluting ions, such as adducts and in-source fragments. Therefore, mz-rt duplicates were handled with care, with potential de-adducting considered via MS-FLO when appropriate. We acknowledge that such degeneracy can lead to an inflated number of features compared to the actual number of metabolites present which is we considered during metabolite identification. 


\paragraph{Blank/extraction-control filtering, intensity thresholds, and sparsity pruning.}
To reduce the background and carryover effects, an extraction control filter was implemented at the feature level. For each feature, the maximum intensity was determined across extraction controls (\(a\)) and biological samples (\(b\)). Features for which \(b < 10\times a\) were eliminated. To preclude the exclusion of borderline yet potentially biological signals, a feature was retained should at least one biological sample exceed the extraction control maximum. Furthermore, a minimum average intensity threshold within the treatment groups of interest (\(\sim 10^{6}\) peak height) was imposed to ensure that downstream analyses would emphasize robust signals. At the sample level, any sample exhibiting \(\geq 70\%\) features as zero (or missing) was excluded prior to normalization and statistical analysis. This pruning of sparsity is essential to prevent unstable scaling and spurious differential signals caused by ultra-sparse profiles. 


\paragraph{MS/MS spectral library matching and cross-referencing of IDs.}
For each feature analyzed by MS/MS, the most intense fragmentation spectrum was queried against the GNPS database. Library matches resulted in putative identifications (levels 2/3), potentially including isomers or near-mass analogs. Features without direct matches were eliminated. To enable quantification with identifications, tables were linked using the feature \emph{row ID} from the MZmine peak list and the corresponding \#Scan\# key in the GNPS results, ensuring a one-to-one correspondence between intensities and candidate identifications. In cases where a feature yielded multiple GNPS hits, a single primary annotation was designated by retaining the highest MQScore (cosine similarity). Ties in the values were resolved based on a greater number of shared fragment ions and a smaller precursor mass error (ppm). All other sub-threshold or lower-ranked candidates were retained for verification but were excluded from subsequent statistical analyses.

\paragraph{Systematic Error Removal Using Random Forest (SERRF).}
Following the cleaning process, the data were then used for SERRF normalization (ref). We used the SERRF server (https://slfan2013.github.io/SERRF-online/\#) to obtain the normalized output. After applying SERRF, only biological samples were preserved. Any zeros were substituted with two-thirds of the minimum nonzero value for that feature to prevent potential infinite logarithmic transformations.

\paragraph{Spatial Correction.}
Finally, we conducted an additional quality control step specifically aimed at eliminating any spatial patterns across our experimental trials. This was achieved using the R package \texttt{SpATS} (ref), which applies a two-dimensional P-spline ANOVA surface over the field coordinates. For every lipid feature, we characterized its intensity as
\begin{align}
  y_{ij} &= \mu + f_{\mathrm{row}}(i) + f_{\mathrm{col}}(j) + f_{\mathrm{row,col}}(i,j) + \varepsilon_{ij},
\end{align}

where 
\begin{itemize}
  \item \(f_{\mathrm{row}}\) and \(f_{\mathrm{col}}\) represent smooth functions that model systematic effects across rows and columns, respectively, 
  \item \(f_{\mathrm{row,col}}\) is a smooth interaction surface that handles more complex spatial gradients. 
}  

We utilized the residuals, defined as the difference between observed intensity and the fitted spatial trend, as our final intensity data. This methodology effectively corrects for positional artifacts, such as edge effects, that could interfere with subsequent analyses. Detailed smoothing parameters, including the number of knots, penalty orders, and comprehensive model specifications, can be found in our GitHub repository at \texttt{scripts/spats\_qc.R}.


\paragraph{Lipid Quantification.}
The identified lipid species were organized into lipid classes and subclasses (see Supplementary Table~1) based on Lipid Maps (ref). In each sample, the total intensity for a class was obtained by summing the intensities of the species within that class. To manage variability in signal intensity due to different runs or injections, these class totals were normalized relative to the total ion current (TIC) of the sample. As a result, the relative abundances were presented as percentages of the TIC by adding up intensities across all lipid classes in the sample. For each class and its subclasses, we determined the TIC fraction for each sample and then averaged these percentages across samples for each condition (Control, $n=384$; LowInput, $n=362$). Lipids were categorized into glycerolipid, glycerophospholipid, sphingolipid, sterol, betaine lipid, fatty acid, ether lipid, and terpenoid. Refer to Supplementary Table 2 for the full list. 

\paragraph{Lipid Ratio Identification.}
To determine the key lipid ratios most significantly influenced by the shift from C to LI, we utilized the cumulative class-level abundances of lipids and calculated all possible pairwise ratios. These calculations were then analyzed through orthogonal partial least squares–discriminant analysis (OPLS-DA) employing the \emph{ropls} package in R (v3.3.2). To identify the most distinguishing ratios, we analyzed the Variable Importance in Projection (VIP) scores generated by OPLS-DA. A threshold of 1 for VIP was used. These high-ranking ratios enhanced the multivariate differentiation between C and LI samples. The model is explained in detail below. 

\paragraph{Lipid Ratio Calculation.}
To quantify condition specific shifts between lipid classes, we used  log\textsubscript{10} ratios of class‐level relative abundances. Normalization proceeded as follows:

\begin{enumerate}
\item \textbf{Per‐sample TIC normalization.}  
For each sample, preprocessed peak intensities were summed (total ion current, TIC) and each species intensity was divided by the sample TIC to yield a relative abundance:
\[
\text{Relative Abundance} = \frac{\text{Intensity}}{\text{TIC}}.
\]

\item \textbf{Log\textsubscript{10} transformation with a pseudo‐count.}  
Because many relative abundances are very small or zero, we added half of the smallest nonzero value in that sample ($\varepsilon$) and computed 
\[
\log_{10}(x + \varepsilon),
\]
to stabilize variance.

\item \textbf{Class‐level aggregation (mean log\textsubscript{10}).}  
Species were grouped into lipid classes (e.g., PC, PE, DGDG, MGDG, TG, DG; see Supp.\ Table~2).  
For each sample $i$ and class $c$, we averaged the species‐level logs:
\[
\text{class\_log}_{i,c} = \frac{1}{n_c} \sum_{k \in c} \log_{10}\!\left(\frac{\text{Intensity}_{i,k}}{\text{TIC}_i} + \varepsilon_i\right).
\]
\end{enumerate}

\paragraph{Ratios (log scale).}
Pairwise \emph{log‐ratios} were computed as differences of class logs, e.g.
\[
\text{DGDG/PC} = \text{class\_log}_{\text{DGDG}} - \text{class\_log}_{\text{PC}} 
= \log_{10}\!\left(\frac{\text{DGDG}}{\text{PC}}\right).
\]
Positive values indicate enrichment of the numerator class (LI $>$ C if the LI–C effect is positive), and negative values indicate the reverse.

\paragraph{Statistical Tests.}
For each ratio, we performed a comparison between LI and C utilizing the two-sample Wilcoxon rank-sum test (Mann–Whitney) on the log ratio values. It was selected for its robustness to unequal group sizes and its ability to handle data with heavy tails without assuming normal distribution. Also, for each ratio, we present the following quantities (SuppTable 5):

\begin{itemize}
\item \texttt{n\_C}, \texttt{n\_LI} = sample sizes in C and LI.
\item \texttt{median\_C}, \texttt{median\_LI} =  group medians of the log‐ratio.
\item \texttt{effect\_log10} = median difference on the log scale, defined as $\text{median}_{LI} - \text{median}_{C}$. Positive means the ratio is higher in LI, negative means the ratio is higher in C. 
\item \texttt{effect\_fc} = fold‐change corresponding to \texttt{effect\_log10}, computed as $10^{\text{effect\_log10}}$ (e.g., $+0.70$ implies $\approx 5.0\times$).
\item \texttt{HL\_low}, \texttt{HL\_high} = 95\% confidence interval for the Hodges–Lehmann (HL) location shift (robust estimate of LI–C difference). If the CI excludes 0, the shift is statistically significant.
\item \texttt{p\_wilcox} = Wilcoxon rank–sum $p$‐value for LI vs.\ C.
\item \texttt{p\_adj\_BH} = Benjamini–Hochberg adjusted $p$‐value (FDR correction across $m=43$ ratios, $\alpha=0.05$).
\item \texttt{AUC\_pct} = probability of superiority (ROC AUC $\times 100\%$), i.e.\ the probability a randomly chosen LI value exceeds a Control value.
\item \texttt{cliffs\_delta} = Cliff’s $\delta$ effect size ($[-1,1]$); $\delta \approx +1$ (or $-1$) indicates nearly complete separation (LI $>$ C or LI $<$ C). 
\item \texttt{jackknife\_stability} = leave‐one‐out sign stability of $\text{median}_{LI} - \text{median}_{C}$ (1.0 means direction invariant to any single sample).
\end{itemize}

\subsection*{Principal Component Analysis (PCA)}  

We performed three complementary PCA workflows in R using \emph{stats} package (ref).  First, we ran PCA on the individual lipid species abundances.  Second, we summed abundances by lipid class (SuppTable1) (e.g. TG, DG, PC, MGDG, SQDG) and repeated PCA to highlight broader shifts in lipids. Third, we computed key log ratios metrics using OPLS-DA (explained below) and carried out PCA on these as well. In all cases, data were mean–centered and scaled prior to analysis. For each PCA, we retained the first two principal components for visualization. 


\subsection*{Genome-wide Association Studies (GWAS)} 

GWAS analyses were carried out for each lipid trait under each condition using the mixed linear model (MLM) featured in GEMMA (v2.3) (ref). To address population structure and relatedness, a centered relatedness matrix (kinship) was computed from SNP genotype data. For each lipid trait, the MLM was applied using the kinship matrix to handle population stratification effects. Besides individual traits, GWAS was also applied to summed lipid classes and all possible ratios (refer to Supplementary Table 1), and the first two PCs of the summed classes. GWAS analysis was conducted on the first two PCs for each class. A significance threshold of \(-\log_{10}(p) \geq 7\) was employed in order to account for multiple comparisons. 
%In contrast, a more lenient threshold of \(-\log_{10}(p) \geq 5\) was applied to all other analyses.

\subsection*{Gene Annotation}

SNPs were aligned with the Sorghum bicolor reference genome v3.1 (BTx623). For each marker, a 50 kb segment was designated, spanning 25 kb on either side, and all gene models within this area were retrieved. Functional annotations and homology were obtained from Phytozome (https://phytozome.jgi.doe.gov), SorghumBase (https://sorghumbase.com), and TAIR for corresponding Arabidopsis thaliana orthologs. Genes with known roles in N, P, cold tolerance, or lipid metabolism were specifically noted. We aggregated the frequency of each candidate gene within all lipid GWAS findings and marked those with the highest recurrence showing -log10(p-values) of 7 or greater.

\subsection*{Orthogonal Projections to Latent Structures–Discriminant Analysis (OPLS–DA)}

OPLS–DA was employed to identify the lipid class ratios that most effectively distinguish between C and LI while reducing unrelated variance. The analysis was confined to ratios from glycero- and glycerophospholipid classes, specifically TG, DG, MG, DGDG, MGDG, PC, LPC, PE, LPE, PA, PS, and PG. All possible pairwise ratios between the mean log relative abundances of classes were computed as explained above. OPLS–DA was conducted using the \texttt{ropls} R package (v1.34.0), wherein the ratio matrix was denoted as $\mathbf{X}$ and the response $Y$ was coded as lipid class. A single predictive component (\texttt{predI = 1}) was defined, whereas the number of orthogonal components was determined through cross-validation (\texttt{orthoI = NA}). The decomposition is:
\[
  \mathbf{X} = T_{p} P_{p}^{T} \;+\; T_{o} P_{o}^{T} \;+\; E,
\]
where,
\begin{itemize}
  \item \(T_{p}\) and \(P_{p}\) are the predictive score and loading matrices capturing variation correlated with \(Y\),
  \item \(T_{o}\) and \(P_{o}\) are the orthogonal score and loading matrices capturing structured variation orthogonal to \(Y\),
  \item \(E\) is the residual matrix representing unexplained variation.
\end{itemize}

To address the issue of overfitting and adjust for differing sample sizes (C: $n=394$, LI: $n=363$), we employed a stratified seven-fold cross-validation approach with folds that are balanced to estimate $R^{2}_{Y}$ (the variance in $Y$ explained) and $Q^{2}$ (cross-validated predictivity). The significance of the model was evaluated using 500 label permutations. One-sided exact $p$-values were derived as $(\#\{\text{perm} \geq \text{obs}\} + 1)/(N_{\text{perm}} + 1)$. Furthermore, we present $R^{2}_{X}$ (the fraction of $X$ variance captured on the predictive axis) to enhance the interpretability of the score plots.

Ratios that discriminate between C and LI were prioritized based on Variable Importance in Projection (VIP) scores, with VIP $> 1$ as the threshold. Since VIP signifies contribution rather than direction, the effect size direction was separately summarized through the calculation of group medians of the log ratios. 

\subsubsection*{Lipid Metabolic Network Analysis (LINEX2)}

We used the Lipid Network Explorer (LINEX2; \url{https://exbio.wzw.tum.de/linex/}) to reconstruct lipid networks and to identify condition enriched subgraphs for C vs. LI, considering both the ratio and difference between them. Lipid names were standardized to align with species notation (class plus acyl composition). LINEX2 constructs a global species network based on curated reaction rules, including headgroup interconversions, (de)acylation/editing, elongation, desaturation, and overlays a quantitative association structure (Spearman correlations across samples) onto reaction edges. For enrichment, LINEX2 calculates substrate–product change scores for each reaction reagrading the C vs. LI ratio and difference, employing a greedy local-search procedure to identify subgraphs that optimize the average substrate and product change. Since, \textit{Sorghum bicolor} is not implemented as a default organism in LINEX2, we selected \textit{Oryza sativa} (OSA) as the reference species for network construction.


\subsubsection*{Lipid Ontology Enrichment and Hierarchical Classification (LION/web)}

We conducted functional ontology for lipids using LION/web (Molenaar \emph{et al.}, 2019; \url{http://www.lipidontology.com}). Lipid nomenclature was standardized according to LIPID MAPS annotations. The default settings of LION/web were applied for enrichment statistics and multiple-testing correction. We retained terms for $q \leq 0.10$. A hierarchical classification analysis of individual lipids and functional ontologies was also performed using the LION/web.


\subsection*{Gradient boosting modeling and module-based SHAP interpretation}

\paragraph{Lipid data processing.}
To achieve variance stabilization and address zero values, each lipid intensity was log\textsubscript{10}-transformed with a pseudocount of 1:
\[
L_{ij}^{\prime} = \log_{10}(L_{ij} + 1),
\]
where $L_{ij}$ is the raw intensity of lipid $j$ in sample $i$. Following transformation, lipid columns were median-centered across samples through feature-wise subtraction of the column median, effectively removing global offsets while maintaining inter-sample variability.

\paragraph{Phenotype and population structure covariates.}
The phenotypic traits for SAP include plant height (PH) and days to anthesis (DTA) (Supp.\ Table~10), which served as response variables. To reduce the likelihood of the model capturing population structure rather than biological associations, both response variables and lipid predictors underwent adjustment through residualization with respect to the first three principal components (PCs), employing ordinary least squares (OLS) regression.

Let $y$ denote the phenotype vector and $X_{\mathrm{PC}}$ the PC design matrix (with intercept). We computed the residual phenotype as
\[
y_{\mathrm{adj}} = y - X_{\mathrm{PC}}(X_{\mathrm{PC}}^{\top}X_{\mathrm{PC}})^{-1}X_{\mathrm{PC}}^{\top}y,
\]
i.e., the residuals from the regression $y \sim X_{\mathrm{PC}}$.
For the lipid matrix $L$ (samples $\times$ features), we removed PC effects feature-wise via the same projection:
\[
L_{\mathrm{adj}} = L - X_{\mathrm{PC}}(X_{\mathrm{PC}}^{\top}X_{\mathrm{PC}})^{-1}X_{\mathrm{PC}}^{\top}L.
\]
The adjusted lipid matrix $L_{\mathrm{adj}}$ and the residual phenotype $y_{\mathrm{adj}}$ were used for all downstream modeling. Importantly, residualization was performed separately within each cross-validation fold to prevent data leakage: coefficients were estimated on training data only and applied to both training and test sets.

\paragraph{Lipid module construction via correlation clustering.}
Due to high correlation among lipid species within the same biochemical class, interpreting SHAP values for individual lipids can be misleading---importance tends to be distributed across correlated features, making it difficult to identify which lipids (or lipid programs) are truly predictive. To address this, we grouped lipids into correlation-based modules prior to interpretation.

We computed the pairwise Spearman correlation matrix across all lipid features and converted it to a distance matrix:
\[
D_{jk} = 1 - |r_{jk}|,
\]
where $r_{jk}$ is the Spearman correlation between lipids $j$ and $k$. Hierarchical clustering (average linkage) was performed on this distance matrix, and modules were identified using the dynamic tree cut algorithm (\texttt{dynamicTreeCut} in R) with \texttt{deepSplit = 2} and a minimum cluster size of 3. Lipids not assigned to any cluster were treated as singleton modules. This procedure yielded 27 lipid modules, each characterized by its dominant lipid class (e.g., TG, PC, DG).

\paragraph{XGBoost model training and cross-validation.}
We employed gradient-boosted trees (XGBoost) to model the PC-residualized phenotype as a function of the adjusted lipid features. XGBoost was chosen over random forests for its typically superior performance on tabular data and its built-in L1/L2 regularization, which encourages sparsity when features are correlated.

The model was trained with the following hyperparameters: learning rate $\eta = 0.1$, maximum tree depth = 6, row subsampling = 0.8, column subsampling = 0.8, L2 regularization $\lambda = 1$, and L1 regularization $\alpha = 0.5$. Models were trained for 150--200 boosting rounds.

To assess generalization performance, we implemented 5-fold cross-validation with PC residualization performed within each fold. For each fold, we computed:
\[
R^2 = 1 - \frac{\sum_{i}(y_i - \hat{y}_i)^2}{\sum_{i}(y_i - \bar{y})^2}, \quad
\mathrm{RMSE} = \sqrt{\frac{1}{n}\sum_{i}(\hat{y}_i - y_i)^2}, \quad
\mathrm{MAE} = \frac{1}{n}\sum_{i}|\hat{y}_i - y_i|.
\]
We report the mean $\pm$ standard deviation of $R^2$ across folds.

\paragraph{SHAP computation and module-level aggregation.}
For model interpretation, a final XGBoost model was trained on all samples (after PC residualization). SHAP (SHapley Additive exPlanations) values were computed using the \texttt{shapviz} package in R, which provides exact TreeSHAP values for gradient-boosted tree ensembles. This yielded an $n \times p$ matrix of SHAP values $\phi_{ij}$, representing the contribution of lipid $j$ to the prediction for sample $i$.

To obtain module-level importance, we aggregated SHAP values within each module. For module $m$ containing lipids $\mathcal{L}_m$, we computed the module SHAP for each sample as:
\[
\Phi_{im} = \sum_{j \in \mathcal{L}_m} \phi_{ij},
\]
and the global module importance as:
\[
\overline{|\Phi|}_m = \frac{1}{n}\sum_{i=1}^{n}|\Phi_{im}|.
\]
Modules were ranked by $\overline{|\Phi|}_m$ in descending order.

\paragraph{Bootstrap stability analysis.}
To assess the stability of module importance rankings, we performed bootstrap resampling (50--100 iterations). In each bootstrap iteration, we: (1) resampled $n$ observations with replacement, (2) trained an XGBoost model on the bootstrap sample, (3) computed SHAP values and module-level aggregations, and (4) recorded the mean absolute module SHAP. From the bootstrap distribution, we computed 95\% confidence intervals for each module's importance. This procedure identifies modules whose importance is robust to sampling variability, as opposed to those whose rankings are unstable.

\paragraph{Cross-dataset and cross-environment validation.}
For plant height, we validated our findings using an independent phenotype dataset from Boyles et al. We computed module SHAP importance separately for each dataset and assessed concordance by: (1) counting the overlap of top-10 modules between datasets, and (2) computing the Pearson correlation of module importance scores across all modules.

For flowering time, we extended the analysis across five environments (our DTA data plus four environments from Wei et al.: MI20, NE20, IA21\_1, IA21\_2). For each environment, we computed module SHAP importance and identified modules that consistently appeared in the top 10 across multiple environments. This cross-environment consistency analysis reveals lipid programs with reproducible associations, even when overall predictive power is limited.
googl


\subsection*{Data Availability}

Data processing and statistical analyzes were performed in R (version 4.3.3) using. All the codes, figures, and pipeline are described in the GitHub repository: github.com/nirwan1265/SoLD\_paper.


\section{Financial Disclosure Statement}

This work was supported by NC State startup funds awarded  
Fieldwork and mapping population development were supported by NSF-PGR award 1546719 
This work is supported by the Research Capacity Fund (HATCH), project award no. 7005660, from the U.S. Department of Agriculture’s National Institute of Food and Agriculture.  
The work on this paper and Nirwan Tandukar was supported by the U.S. Department of Energy, Office of Science, Biological and Environmental Research program, Early Career Award Number DE-SC0021889.
Allison Barnes was supported by NSF-PGRP PRFB grant 2010703. 
Fausto Rodríguez-Zapata was supported by the Science and Technologies for Phosphorus Sustainability (STEPS) Center, a National Science Foundation Science and Technology Center (CBET-2019435).
This work was performed in part by the Molecular Education, Technology and Research Innovation Center (METRIC) at NC State University, which is supported by the State of North Carolina. 

\section{Acknowledgments}
We thank the  Puerto Vallarta Winter Nursery crews who have helped generate introgression populations used in this manuscript.
We especially want to acknowledge the indigenous people of the Americas and the ingenuity with which they domesticated and facilitated the spread and adaptation of maize throughout the continent.
This work would not have been possible without the international maize research community and the willingness of so many colleagues to support the development of new research programs.
Any opinions, findings, conclusions, or recommendations expressed in this publication are those of the author(s) and should not be construed to represent any official USDA, NSF, DOE, ARS or U.S. Government determination or policy.

\bibliography{Inv4mPhosphorus}

\pagebreak

\onecolumn


\section*{Supplement}
\beginsupplement

% =========================
% SUPP FIGURE 1
% =========================
\begin{figure*}[!ht]
  \centering
  \includegraphics[width=\textwidth]{figures/supp/SuppFig_S1_QC_RunOrder_SERRF_PCA_SpATS.png}
  \caption{\textbf{QC, SERRF normalization, PCA tightening, and SpATS spatial residual diagnostics.}}
  \label{fig:suppfig_s1_qc}
\end{figure*}

% =========================
% SUPP FIGURE 2
% =========================
\begin{figure*}[!ht]
  \centering
  \includegraphics[width=\textwidth]{figures/supp/SuppFig_S2_Compositional_Contrasts.png}
  \caption{\textbf{Class composition and compositional contrasts.}}
  \label{fig:suppfig_s2_composition}
\end{figure*}

% =========================
% SUPP FIGURE 3
% =========================
\begin{figure*}[!ht]
  \centering
  \includegraphics[width=\textwidth]{figures/supp/SuppFig_S3_Lipid_Species_Counts.png}
  \caption{\textbf{Lipid species counts, overlap, and superclass summaries.}}
  \label{fig:suppfig_s3_counts}
\end{figure*}

% =========================
% SUPP FIGURE 4
% =========================
\begin{figure*}[!ht]
  \centering
  \includegraphics[width=\textwidth]{figures/supp/SuppFig_S4_PCA_Lipids.png}
  \caption{\textbf{Lipid PCA in Control and LowInput showing class-structured clustering.}}
  \label{fig:suppfig_s4_pca}
\end{figure*}

% =========================
% SUPP FIGURE 5
% =========================
\begin{figure*}[!ht]
  \centering
  \includegraphics[width=\textwidth]{figures/supp/SuppFig_S5_TopVariance_Lipids.png}
  \caption{\textbf{Top high-variance lipids across genotypes in Control and LowInput.}}
  \label{fig:suppfig_s5_variance}
\end{figure*}

\clearpage


\paragraph*{S1 File.}
\phantomsection
\makeatletter
\def\@currentlabelname{S1 File.}
\makeatother
\label{S1_File}
\textbf{High Confidence Senescence Associated DEGs.} High Confidence DEGs that have been reported to be associated with senescence, they might respond to any of the experimental predictors in this study: -P, Leaf, \invfour genotype.

\end{document}


